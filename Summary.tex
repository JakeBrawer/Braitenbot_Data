% Created 2015-09-14 Mon 12:53
\documentclass[11pt]{article}
\usepackage[utf8]{inputenc}
\usepackage[T1]{fontenc}
\usepackage{fixltx2e}
\usepackage{graphicx}
\usepackage{grffile}
\usepackage{longtable}
\usepackage{wrapfig}
\usepackage{rotating}
\usepackage[normalem]{ulem}
\usepackage{amsmath}
\usepackage{textcomp}
\usepackage{amssymb}
\usepackage{capt-of}
\usepackage{hyperref}
\usepackage{setspace}
\doublespacing
\usepackage[margin=2.54cm]{geometry}
\author{Jake Brawer}
\date{\today}
\title{Thesis Summary}
\hypersetup{
 pdfauthor={Jake Brawer},
 pdftitle={Thesis Summary},
 pdfkeywords={},
 pdfsubject={},
 pdfcreator={Emacs 24.5.1 (Org mode 8.3.1)}, 
 pdflang={English}}
\begin{document}

\maketitle
Folk conceptions of genotype-phenotype mapping functions (G-P maps) err towards the bijective. 
That is, there is a tendency to view any given gene as having a correspondence with one, and only one 
trait in a given organism. However, there is an overwhelming amount of evidence suggesting that such a view is completely 
antithetical to biological reality (Oyama, 1985). What appears to be the case is that, across most organisms,
 phenotypes are the result of a complex and multi-leveled interaction not only between multiple genes, but between genetic and ontogenetic factors as well. 
This  interplay is evident from the observation that traits will only manifest if certain genetic factors are present in
 concert with certain environmental factors ( both intrinsic and extrinsic to the organism). Such “constructive interactionism” 
is observed at nearly all levels of ontogeny, from zygotic development, e.g. the contribution of the maternal environment to
 embryonic development (Oyama, 1985) through the organism’s lifetime, e.g. the diathesis-stress model of psychopathology (Walker, 1997).
 Biological agents, then, are not the products of rigid instruction sets or stochastic causalities, but are highly dynamical beings,
 subject to a fantastically complex process of co-action between genes and environment, a process that defies our 
 most heroic parsing efforts.\\
\indent For  a little over a year, Aaron Hill ('16) and I have been working on a system for evolving the Johuco Ana Bbots (Braitenbots), owned by the 
Cognitive Science department, that incorporates some of the concepts discussed above. The Braitenbots are essentially analog computers on wheels, with 
the behavior of each robot determined by easily mutable wire connections that connect components of the computer. We have developed a way to encode
these wire connections into binary "genomes," which can be then be transmitted to successive generations via a processes that models some of the genetic processes
underlying sexual reproduction (i.e. genetic recombination). The mapping of a gene (or genes) to a trait (or traits) is determined by the rules used by
the system to decode and instantiate the genotype (our analogue for development), thereby limiting arbitrary and \emph{a priori} mapping schemes. Preliminary experiments in simulation
have confirmed that cross-generational, adaptive evolution is possible using this model. With this in mind I would like to use the model to run a number of experiments
 probing the mechanics of evolution. More specifically I am interested in delineating the relationship between robustness and evolvability.
 Both the size of intronic regions (Soule, 2003), and the frequency of recombination events (Hu, Banzhaf, and Moore, 2014) have been shown to facilitate phenotypic robustness 
and thus adaptive evolution. By running populations of Braitenbots with different parameters, i.e. intron size, recombination point location and distribution, etc. through 
evolutionary trials, I can evaluate this putative robustness-evolvability relationship. 

\section*{References}
\label{sec:orgheadline1}
Hu, T., Banzhaf, W., \& Moore, J. H. (2014). \emph{The effects of recombination on phenotypic exploration and robustness in evolution.} Artificial life.\\
\indent Explores the effects of recombination on phenotypic variation in genetic algorithmms. Finds that recombination is not only less disruptive than mutation,
but promotes the development of robust phenotypes, due in part to the fact that reorganization of genetic components (re: recombination) is generally
less deleterious than mutational effects.\\
Oyama, S. (1985). \emph{The ontogeny of information: Developmental systems and evolution}. Cambridge: Cambridge University Press.\\
\indent Discusses the false dichotomy between genetic and developmental systems. Argues for a "constructive interactionist" approach to development.
rather than the old genocentric models. Highly influenced the design of this model. \\
Soule, T. (2003). \emph{Operator choice and the evolution of robust solutions.} In Genetic programming theory and practice (pp. 257-269). Springer US.\\
\indent Discusses code bloat, the increase in size of intronic regions. Argues that this processes increases the robustness of a
gene and thus is adaptive.\\
Walker, E. F., \& Diforio, D. (1997). \emph{Schizophrenia: a neural diathesis-stress model.} Psychological review, 104(4), 667. \\
\indent Argues for a diathesis-stress model of schizophrenia, i.e., that both developmental factors (e.g. psychosocial stressors) and 
genetic factors contribute to schizophrenic symptomatology. A great example of constructive interactionism.\\
\end{document}